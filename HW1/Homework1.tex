\documentclass{article}
\usepackage[utf8]{inputenc}

\title{The Irrationality of $\sqrt{2}$}
\author{Benny Chen}
\date{September 2, 2022}

\usepackage{color}
\usepackage{amsthm}
\usepackage{amssymb} 
\usepackage{amsmath}
\usepackage{listings}
\usepackage{xcolor}
\usepackage{listings}
\usepackage[hidelinks]{hyperref}

\newtheorem{Definition}{Definition}
\newtheorem{Theorem}{Theorem}
\begin{document}

\maketitle

\section{Introduction}
In this short paper we will show that $\sqrt{2}$ is an irrational number. 
In \hyperref[sec:Preliminaries]{Section 2} (use hyperref so section numbers in text are clickable), we 
will introduce the concepts of rational and irrational numbers. In Section 3, 
we will provide a proof of the fact that $\sqrt{2}$ is irrational. Finally, in 
Section 4, we will discuss what we know about other numbers that are known to be irrational.

\section{Preliminaries}
\label{sec:Preliminaries}
First, we define the rational numbers.
\begin{Definition}
Let $\mathbb{Q}$ be all rational numbers. A rational number is any real number that can be written as a 
ratio such as a quotient or a fraction ($\frac{p}{q}$, $q$ being a non zero number) of two integers.


\end{Definition}
And now we can define the cocept of irrationality.
\begin{Definition}
A rational number is any real number that is not a quotient or a fraction of two integers. The number
would be a infinitely continuous number without repeating.
\end{Definition}

\section{The Proof}
We are finally ready to show our main result
\begin{Theorem}
    The real number $\sqrt{2}$ is irrational\\
    Proof. 
\end{Theorem}
Assume that $\alpha = \sqrt{2}$ and $\alpha^{2} = 2$. Now suppose that $\alpha = \frac{m}{n}$ where
$m, n$ are in their lowest terms, meaning that $m$ and $n$ are coprime with only 1 being their only factor. 
We can now also surmise that $(\frac{m}{n})^2 = 2$ so that $m^2 = 2n^2$. Both $m$ and $n$ can be divided divided by 2 as 
$m$ is divided by $2n^2$ which will be a even number. 

\section{Irrational Number}
There are many example of numbers that deviate from being rational, making them irrational.
More famous examples are $\pi$ and Euler's Number. 
\\\\$\pi$ is one of the most famous and well known numbers of all time. $\pi$ is the ratio of 
the circumference of any circle to the diameter of that circle. $\pi$ was first realized by
Greek mathematician Archimedes of Syracuse around 250 B.C who created a algorithm to approximate
the number called Archimedes' constant. The number approximated by Archimedes, $\pi$, is also defined as
irrational as the decimal continuously goes to infinity. There have been many proofs done on the subject
that show that $\pi$ is irrational from Lambert's proof to Hermite's proof.
\\\\Euler's Number is also a famous irrational number. Euler's number is a important mathematical constant
that is the base of the natural logarithm. Euler's number was first discovered by Jacob Bernoulli in the late 1600's
but was not known until the 17th century which when it was named after Swiss mathematician Leonhard Euler.
Euler's number is similar to $\pi$ in that it is irrational and has a continuous decimal expansion. There have been many proofs
done that support that it is irrational like Euler's proof and Fourier's proof.

\begin{thebibliography}{9}
    \bibitem{website:https://www.britannica.com/science/pi-mathematics}
    Britannica, The Editors of Encyclopaedia. "pi". Encyclopedia Britannica, 14 Mar. 2022, https://www.britannica.com/science/pi-mathematics. 
    
    \bibitem{website:https://mathshistory.st-andrews.ac.uk/HistTopics/e/}
    https://mathshistory.st-andrews.ac.uk/HistTopics/e/
\end{thebibliography}

\end{document}