\documentclass[12pt]{article}
\usepackage[utf8]{inputenc}

\title{The Quadratic Formula}
\author{Benny Chen}
\date{September 15, 2022}

\usepackage{color}
\usepackage{amsthm}
\usepackage{amssymb} 
\usepackage{amsmath}
\usepackage{listings}
\usepackage{xcolor}
\usepackage{listings}
\usepackage[hidelinks]{hyperref}

\newtheorem{Definition}{Definition}
\newtheorem{Theorem}{Theorem}
\begin{document}

\maketitle
 
\section{Introduction}
In this short paper we will show that the quadratic formula is a useful tool for solving algebraic equations and ultimately
finding the roots of the equation. In \hyperref[sec:Background]{Section 2}, we will introduce the concept and background of the quadratic equations. In Section 3, we will define what the quadratic formula is and how it is used to solve quadratic equations. In section 4, we will provide a proof of the fact that the quadratic formula 
is a useful tool for solving quadratic equations. In \hyperref[sec:Examples]{Section 5} we will 
give an examples of how to use the quadratic formula to solve a quadratic equation.

\section{Background}
\label{sec:Background}
A quadratic equation is an equation that can be written in the form $ax^2 + bx + c = 0$ where $a, b, c$ are real numbers and 
$a \neq 0$. The quadratic formula is a formula that can be used to solve a quadratic equation. The quadratic formula is:
\[x = \frac{-b \pm \sqrt{b^2 - 4ac}}{2a}\]
The formula can be used to solve a quadratic equation by substituting the values of
$a, b, c$ into the quadratic formula. The quadratic formula has been used for centuries to solve problems. Most speculate that
ancient Babylonians were the first to use the quadratic formula. They had no real concept of an equation, but did create an 
algebraic approach to solve problems. The mathematical approach finally started beginning to take form with the Greeks which 
lead to 300BC when Euclid started to write down theorems and proofs. However, the quadratic formula was not fully 
developed until the 16th century when the Italian mathematician Gerolamo Cardano developed the formula. The cases for 
the quadratic formula were then covered by Simon Stevin in 1585. The formula has been used ever since and is still used
today. \\\\

\section{Preliminaries}
\label{sec:Preliminaries}
First, we define the quadratic equation.
\begin{Theorem}
Let $ax^2 + bx + c = 0$ be a quadratic equation. A quadratic equation is an equation that can be written in the form $ax^2 + bx + c = 0$ where $a, b, c$ are real numbers and
$a \neq 0$.
\end{Theorem}
Next, we define the quadratic formula.
\begin{Theorem}
Let $ax^2 + bx + c = 0$ be a quadratic equation. The quadratic formula is a formula that can be used to solve a quadratic equation and finds its roots. The quadratic formula is:
\[x = \frac{-b \pm \sqrt{b^2 - 4ac}}{2a}\]
\end{Theorem}

\section{The Proof}
\label{sec:Proof}
\begin{Theorem}
We can prove that the quadratic formula can be used to find the roots of a quadratic equation by 
deriving what $x$ is. The quadratic equation can be written as
\[ax^2 + bx + c = 0.\]
We first divide both sides by $a$ to get
\[\frac{ax^2}{a} + \frac{bx}{a} + \frac{c}{a} = 0.\]
Then we can subtract $\frac{c}{a}$ from both sides to get
\[\frac{ax^2}{a} + \frac{bx}{a} = -\frac{c}{a}.\]
We can then complete the square by adding $\frac{b^2}{4a}$ to both sides to get
\[\frac{ax^2}{a} + \frac{bx}{a} + \frac{b^2}{4a} = -\frac{c}{a} + \frac{b^2}{4a}.\]
We can simplify both sides down to
\[\left(x + \frac{b}{2a}\right)^2 = \frac{b^2-4ac}{4a^2}.\]  

We can now take the square root of both sides to get
\[x + \frac{b}{2a} = \pm \frac{\sqrt{b^2-4ac}}{2a}.\]
We can now subtract $\frac{b}{2a}$ from both sides to get
\[x = \pm \frac{\sqrt{b^2-4ac}}{2a} - \frac{b}{2a}.\]
We can now simplify the equation to get
\[x = \frac{-b \pm \sqrt{b^2 - 4ac}}{2a}.\]
We can see that the quadratic formula is the same as the equation we derived.
We can now use this value of $x$ to solve the quadratic equation. 
Therefore, we can conclude that the quadratic
formula can be used to solve a quadratic equation and ultimately find the roots of the equation. \\

\end{Theorem}


\section{Examples}
\label{sec:Examples}
We can use the quadratic formula to solve a quadratic equation. We will use the quadratic equation $x^2 + 2x + 1 = 0$ to
solve for the roots. We can substitute the values of $a = 1, b = 2, c = 1$ into the quadratic formula to get
\[x = \frac{-2 \pm \sqrt{4 - 4}}{2}.\]
We can now simplify the equation to get
\[x = -1.\]

An edge case of the quadratic formula is when the equation does not have a real solution. We will use the quadratic equation
$x^2 + 2x + 3 = 0$ to solve for $x$. We can substitute the values of $a = 1, b = 2, c = 3$ into the quadratic formula to get
\[x = \frac{-2 \pm \sqrt{4 - 12}}{2}.\]
We can now simplify the equation to get
\[x = \frac{-2 \pm \sqrt{-8}}{2}.\]
The equation does not have a real solution because $\sqrt{-8}$ is not a real number. 


\end{document}